\documentclass[a4paper, 12pt]{scrreprt}

\usepackage[utf8]{inputenc}
\usepackage[ngerman]{babel}
\usepackage[T1]{fontenc}
\usepackage{amsmath}
\usepackage[onehalfspacing]{setspace}
\bibliographystyle{unsrt}

\title{Das IEEE 802.15.4 MAC Protocol}
\author{Philipp Lange}
\date{13. Juli 2017}

\begin{document}

\maketitle
\tableofcontents

\newpage
\chapter{Einleitung}
Im Laufe der letzten Jahre wurde die Technik für elektronische Geräte immer günstiger in der Herstellung und auch die Größe eben dieser konnte immer weiter reduziert werden. Diese Faktoren sorgen auch dafür, dass der Fokus nicht nur wie bisher, auf Mobiltelefonen und Computer beim Thema Vernetzung von elektronischen Geräten liegt, sondern immer mehr Sensoren und sogar Haushaltsgeräte miteinander kommunizieren können.
Um alle erdenklichen Geräte mit einander kommunizieren lassen zu können, erfordert es allerdings einen Standard, der Regeln für den Informationsaustausch zwischen ihnen vorgibt. Ohne diese Festlegungen ist es nicht möglich unabhängig von der Art des Gerätes oder dem Hersteller eine Verbindung herzustellen.
In dieser Arbeit soll es um diesen angesprochenen IEEE Standard gehen. Im speziellen wird das Augenmerk auf dem MAC Protokoll liegen. \cite{Karl2007}.

\chapter{Der IEEE 802.15.4 Standard}
Das IEEE 802.15.4 Protocol wurde in der ersten Version im Oktober 2003 veröffentlicht. Die aktuelle, dritte Version gibt es seit Ende 2015. 
Ziel bei der Entwicklung des IEEE 802.15.4 Protocols war es einen einfachen Standard zu entwickeln, die eine Datenübertragung bei sehr niedriger Leistungsaufnahme und die Verwendung kostengünstiger Bauteile ermöglicht. Es gab vorher bereits die Standards IEEE 802.11 und Bluetooth. Diese sind jedoch verhältnismäßig sehr komplex und auch der Energiebedarf ist höher als bei dem IEEE 802.15.4  



\newpage
\bibliography{literatur}
\end{document}
